\documentclass[letterpaper, 11pt]{report}
% unused documentclass options:
% twocolumn

\usepackage{amsmath}
\numberwithin{equation}{section}
\usepackage[nodayofweek,us]{datetime}

\usepackage{geometry}
\geometry{
	letterpaper,
	top=1in,
	bottom=1in,
	left=1in,
	right=1in
}

\begin{document}
	
	
	\author{Ryan Jensen}
	\title{Math Problems}
	\date{\today}
	\maketitle
	
	
	
	\tableofcontents
	
	
	
	
	
	
	\chapter{Partial Sums}
		
		
		
		\section{Partial Sum Order 0}
			
			Find a closed-form expression in $ n $ such that
			
			\begin{equation}
				f(n) = 1 + 1 + 1 + \cdots + 1
			\end{equation}
			
			Note: in the above equation, there are $n$ 1's.
			
		\section{Partial Sum Order 1}
				
			Find a closed-form expression in $ n $ such that
			
			\begin{equation}
				f(n) = 1 + 2 + 3 + \cdots + n
			\end{equation}
			
			
		\section{Partial Sum Order 2}
			
			Find a closed-form expression in $ n $ such that
			
			\begin{equation}
				f(n) = 1^2 + 2^2 + 3^2 + \cdots + n^2
			\end{equation}
			
			
		\section{Partial Sum Order $n$}
			
			Find a closed-form expression in $ n $ such that
		
			\begin{equation}
				f(n) = 1^n + 2^n + 3^n + \cdots + n^n
			\end{equation}
			
			
		\section{Partial Sum of $ n^n $}
			
			Find a closed-form expression in $ n $ such that
			
			\begin{equation}
				f(n) = 1^1 + 2^2 + 3^3 + \cdots + n^n
			\end{equation}
	
	
	
	
	\chapter{Gary Larson's Recommended Problems}
		
		
		
		\section{Hypatia's Problem}
			
			Given two integers \(a\) and \(b\), find integer values for \(x\) and \(y\) such that
			
			\begin{equation}
				(x-y) = a
			\end{equation}
			
			\begin{equation}
				x^2 - y^2 = (x-y) + b
			\end{equation}
			
			
		\section{Ramp}
			
			Describe a ramp such that a ball will reach the bottom in the same amount of time no matter where its starting position is.
			
			
		\section{Factor a 15th Order Polynomial}
			
			Factor the polynomial
			
			\begin{equation}
				x^{15} + 1
			\end{equation}
			
			into two polynomials of order 6 and order 9, like so:
			
			\begin{equation}
				(a_6x^6 + a_5x^5 + \cdots + a_0)(b_9x^9 + b_8x^8 + \cdots + b_0)
			\end{equation}
			
		
	
	
\end{document}